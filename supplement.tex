\documentclass[]{article}
\usepackage[round]{natbib}

\usepackage{fullpage}
\usepackage{listings}
\usepackage{url}
\usepackage{authblk}
\usepackage{graphicx}
\usepackage{color}
\usepackage{booktabs}

% lorem ipsum dummy text
\usepackage{lipsum}

\lstset{language=Python}

% cross-reference with main text
\usepackage{xr}
\externaldocument{paper}

% local definitions
\newcommand{\sgcomment}[1]{{\textcolor{red}{SG: #1}}}
\newcommand{\bmhcomment}[1]{{\textcolor{blue}{BMH: #1}}}
\newcommand{\aprcomment}[1]{{\textcolor{magenta}{APR: #1}}}
\newcommand{\tdwcomment}[1]{{\textcolor{cyan}{TDW: #1}}}

\usepackage{array}
\newcolumntype{P}[1]{>{\raggedright\arraybackslash}p{#1}}

\begin{document}
\title{Supporting Information for\\
``A weakly structured stem for human origins in Africa''}
\author{Aaron P. Ragsdale, Timothy D. Weaver, Brenna M. Henn, and Simon Gravel}
\date{\today}
\maketitle

\renewcommand{\thefigure}{S\arabic{figure}}
\renewcommand{\thetable}{S\arabic{table}}
\renewcommand{\theequation}{S\arabic{equation}}
\setcounter{figure}{0}
\setcounter{table}{0}
\setcounter{equation}{0}

\tableofcontents
\newpage

\section{Data and sequencing}

\subsection{Sequencing and variant calling}

Low coverage (4-8x) Illumina short read data were generated for the Nama,
Gumuz, Amhara and Oromo populations as part of the African Diversity Reference
Panel (Sanger / Wellcome Trust) \citep{Gurdasani2015-qy,Pagani2015-pz} and
approved through a secondary data analysis agreement for this project. Briefly,
raw reads were aligned to GRCH37 with BWAmem \aprcomment{cite}, duplicates
marked with Picard MarkDuplicates, reads were realigned around indels with GATK
RealignerTargetCreater / IndelRealigner followed by BQSR with dbSNP 137.
Contamination checks were performed requiring that FREEMIX $<0.05$;
contamination checks resulted in the elimination of 22 Nama samples. We note
that the high heterozygosity in these genomes both due to inherent genetic
diversity and admixture may have violated base assumptions for this
heterozygosity check. Genomes were then variant called with GATK3.2 Unified
Genotyper \aprcomment{cite} using joint calling across 2,478 individuals within
the ADRP dataset with a minimum base quality of 17. Data were merged with 1000
Genomes Phase 3 \citep{1000_Genomes_Project_Consortium2015-zq} using the union
of sites identified with bcftools isec (-n+1) \aprcomment{cite} then refined
with Variant Recalibrator with a truth sensitivity threshold of 99.5\%. HapMap
III and dbSNP 138 served as known sites while 1000 Genomes Phase 1 Omni2.5 and
Phase 1 genomic SNP served as the training set. After VR, no batch effects were
observed along PC1 and PC2 for 1000 Genomes vs. ADRP. Phasing on the combined
dataset was performed via SHAPEIT2 \aprcomment{cite} and utilized the duoHMM
option for duos and trios. We highlight 82 Nama genomes which are newly
available (unpublished) under accession number EGAD00001006198. Among these 82
Nama samples, we down-sampled the individuals to minimize close relatives and
admixture, such that 44 Nama genomes were retained. 2nd and 3rd degrees
relatives were inferred from reconstructed pedigrees with Omni2.5 SNP array
data. Individuals with $>70\%$ estimated Khoe-San ancestry were retained for
analysis, after partitioning ancestry into $k=6$ clusters with ADMIXTURE
\aprcomment{cite} where alternative ancestries represent European, West
African, Near Eastern, and East African gene flow. 

\subsection{Nama sample collection and consent}

DNA samples were collected in South Africa (Richtersveld) with IRB approval
from Stanford University, Stellenbosch University (ZA) and SUNY Stony Brook.
\aprcomment{Does this need to be expanded on at all?}

\subsection{Details about populations used in analyses}

\sgcomment{To be expanded. TODO: Brenna}
\begin{itemize}
    \item From the merged dataset of the African Diversity Reference Panel and
        1000 Genomes phase 3 data, subsampled to population included in this study
    \item West Africa: Mende from Sierra Leone (MSL); South Africa: newly sequenced 
        Nama; East Africa: Gumuz (traditionally hunter-gatherer with low levels of 
        Eurasian admixture), Oromo and Amhara (combined as traditionally 
        agriculturalists with large proportion of back-to-Africa Eurasian ancestry); 
        Eurasian: British (GBR)
    \item Combined with the high coverage Vindija neanderthal genome
    \item For running Relate, we used a larger set of populations. We kept all 
        African 1000 Genomes populations, along with the Nama, Gumuz, Oromo, and
        Amhara, as well as 4 Eurasian populations from 1000 Genomes: GBR, CEU, PJL,
        and CHB
\end{itemize}

\subsection{Filtering and subsetting data}

All analyses presented in this work focus on biallelic single nucleotide
polymorphisms within the 1000 Genomes Phase 3 strict mask.

For the moments-LD analysis, we focused on intergenic locations because these
appear less affected by natural selection compared to both synonymous and
nonsynonymous variation \citep{Ragsdale2018-dd}. To enable comparison with
Neanderthal DNA, we excluded regions for which the Vindija Neanderthal sample
had less than 100 contiguous base pairs. 

\section{Computing statistics}

\subsection{LD and diversity statistics used in model fits}

\aprcomment{$D^2$, $D_z$, $\pi_2$, and $H$. Definitions, notation,
indexing\ldots}

\subsection{Computing LD and diversity statistics}

We compared single- and two-locus statistics in the data to predictions based
on detailed demographic models. Model predictions were obtained using
recursions described in \citet{Ragsdale2019-nt} and implemented in the software
\emph{moments.LD} (\url{https://bitbucket.org/simongravel/moments/src/main/}).
The model computes expected patterns of single-nucleotide pairwise diversity
and linkage disequilibrium as a function of recombination distance between
variants within and across populations, under the assumption of neutrality.

For numerical convenience, observed genetic variants were binned by
recombination distances. We assessed the robustness of the statistics to errors
in the recombination maps by considering two different recombination maps, the
OMNI YRI and HapMapII
\citep{1000_Genomes_Project_Consortium2015-zq,International_HapMap_Consortium2007-vn}.
Statistics were largely unchanged by using a different map
(Figure~\ref{fig:sup-map_comparison}). \aprcomment{These maps are both inferred
using array data, which is sparse. Comment on this.}

We removed bins of recombination distance less than a recombination distance of
$r = 5\times10^{-6}$ (at a rough estimate of 1 cM/Mb, this corresponds to a minimum
distance of 500 bp on average) to avoid previously reported biases at short
distances due to processes like multinucleotide mutations
\citep{Harris2014-zg,Ragsdale2019-nt}. To avoid uncertainty in phasing, we used
unphased genotypes to compute LD statistics, as proposed in
\citet{Ragsdale2020-nz}. 

Finally, we estimated uncertainty due to the finite amount of genetic material
used in inference using bootstrap over 500 segments along the genome with
roughly equal lengths of retained sequences within each segment. First, for
each distance bin, we used these bootstrap samples to obtain a
variance-covariances matrix across all statistics. This variance-covariance
matrix was used to obtain a model likelihood for each recombination distance
bin and single-locus nucleotide diversity, as a multivariate Gaussian
likelihood. The full model likelihood was taken as the product of likelihoods
over each bin. In other words, we optimized a composite likelihood where
observations in different bins were taken to be independent. To account for
correlations across bins in uncertainty estimates, we estimated parameter
confidence intervals using the same bootstrap set using the Godambe information
matrix \citep{Coffman2016-yq}. \sgcomment{is redundancy here with section
"Optimization using moments". I think we can get rid of this and say:
"optimization and uncertainty calculations are described in section
"Optimization using moments""?, This may be a bit more complicated, since the
Optimizatino section relies on this description. I think we should just punt
the bootstrap description to that section. TODO: discuss Aaron + Simon}

\subsection{Estimating two-locus statistics with small sample sizes}

The approach from \citet{Ragsdale2020-nz} provides unbiased estimates of the LD
statistics considered here, with smaller sample sizes causing greater
uncertainty in the estimated statistics, but is accounted for by computing
variances/covariances via bootstrap. \sgcomment{I don't think that this is true
if our bootstrap is over genomic regions. In an infinite genome with a tiny
sample size, we would estimate no uncertainty...  Or at least, it is only true
if the Neandertal form a truly randomly mating population...  }

Some statistics, such as $D^2$, require at least two diploid samples to
compute. Since we used a single Neanderthal sample, these statistics for the
Neandertal population were not used in the fit. By contrast, there are
statistics that only require a single sample per population to estimate. These
include cross-population heterozygosity, as well as some statistics involving
more than one population. For example, statistics of the form
$D_{human}(1-2p_{human})(1-2q_{neanderthal})$ require a single Neanderthal
sample and are informative of the Neanderthal demography. These statistics were
included in the fit, but statistics requiring more than one Neanderthal sample
to estimate were removed.

\subsection{Computing conditional SFS}

The conditional site frequency spectrum (or cSFS) is the distribution of allele
frequencies restricted to loci that satisfy a given condition. Specifically, we
consider the distribution of allele frequencies in present-day populations
conditioned on the Vindija Neanderthal carrying the derived allele relative to
the inferred ancestral allele. Ancestral alleles alleles were determined from a
6 primate alignment \aprcomment{cite}. This cSFS is expected to be close to
uniform under neutrality and a simple split model (with no subsequent
migration) between the ancestors of modern humans and Neanderthal
\citep{Chen2007-iy}. By contrast, a U-shaped distribution has been taken as
evidence for archaic introgression from a population whose split from modern
humans is at least as old as that of Neanderthal’s
\citet{Durvasula2020-td,Yang2012-ze}. Because we wanted to compare our
inferences (based on intergenic sites) to inferences from previous work (based
on whole-genome data), we computed the cSFS for both intergenic and all sites
genome-wide. Sites with no calls in the Vindija Neanderthal were excluded from
this analysis. \aprcomment{Figures XX.}

Because we were concerned that cSFS analyses may be affected by incorrect
inference of the ancestral allele \cite{Hernandez2007-mf}, we computed the cSFS
for all mutations, and for transitions and transversions separately
\aprcomment{(Figure X, cSFS)}. Comparisons of these observed cSFS with model
predictions are discussed in the model prediction section below. 

\section{Model specification and fitting}



\subsection{General strategy for building models and introducing complexity}

\subsection{Optimization using moments}

\subsection{Confidence intervals using Godambe methods}

\section{Gene genealogy reconstruction}

\section{Predictions from inferred demographic models}

\subsection{$F_{ST}$ between coexisting populations over time}

\subsection{$f_4$ statistics between pairs of contemporary and pairs of ancient populations}

\section{Validations using simulatinos from inferred demographic models}

\subsection{Simulation details}

\subsection{cSFS prediction under inferred models}

\subsection{Relate curves from inferred models}

\subsection{Distribution of deep branch affinities to Neanderthal sequence}

\subsection{Mutation versus recombination rates}

\break

\addcontentsline{toc}{section}{References}
\bibliographystyle{genetics}
\bibliography{paper}

\clearpage

\addcontentsline{toc}{section}{Supporting tables}
\section*{Supporting tables}

\begin{table}[ht]
\caption{
    \label{tab:single_origin}
    \textbf{Best-fit parameters from the Single-Origin model.}
    \aprcomment{fill in caption - generation time of 29 years, other details}
}
\centering
\begin{tabular}[t]{rP{8cm}cc}
    \toprule
    Parameter & Description & Value & Std. err.\\
    \midrule
    $N_e$ & Ancestral effective population size & 10198 & 403 \\
    $N_{MH}$ & Size of modern-human lineage between Neanderthal and Nama splits & 21111 & 529 \\
    $N_{Nama_0}$ & Initial Nama size & 10224 & 370 \\
    $N_{Nama_F}$ & Final Nama size & 222 & 9 \\
    $N_{MSL_0}$ & Initial Mende size & 17211 & 769 \\
    $N_{MSL_F}$ & Final Mende size & 16822 & 606 \\
    $N_{EA}$ & Size of East African branch & 7139 & 273 \\
    $N_{Gumuz_F}$ & Final Gumuz size & 3831 & 131 \\
    $N_{EP}$ & East African agriculturist size & 13033 & 491 \\
    $N_{GBR_0}$ & Initial British size & 846 & 33 \\
    $N_{GBR_F}$ & Final British size & 12121 & 507 \\
    $N_{Neand}$ & Neanderthal size & 1867 & 105 \\
    $T_{Nama}$ & Nama split time (years) & 110400 & 2525 \\
    $m_{Nama-MSL}$ & Nama--Mende symmetric migration rate & $2.82\times10^{-5}$ & $0.158\times10^{-5}$ \\
    $m_{Nama-EA}$ & Nama--East Africa symmetric migration rate & $4.94\times10^{-5}$ & $0.197\times10^{-5}$ \\
    $m_{MSL-EA}$ & Mende--East Africa migration rate & $18.76\times10^{-5}$ & $0.764\times10^{-5}$ \\
    $m_{EA-GBR}$ & East Africa--Europe migration rate & $4.42\times10^{-5}$ & $0.239\times10^{-5}$ \\
    $m_{EA-EA}$ & Intra-East Africa migration rate & $41.28\times10^{-5}$ & $1.33\times10^{-5}$ \\
    $f_{GBR \rightarrow EP}$ & Ancestry proportion of East African agriculturalists from GBR 12 ka ($1-f$ from Gumuz) & 0.658 & 0.0039 \\
    $f_{EP \rightarrow Nama}$ & Ancestry proportion from EA pastoralists to Nama 2 ka & 0.279 & 0.0039 \\
    $f_{GBR \rightarrow Nama}$ & Ancestry proportion from Europeans to Nama 10 generations ago & 0.150 & 0.0019 \\
    \bottomrule
\end{tabular}
\end{table}

\begin{table}[ht]
\caption{
    \label{tab:continuous_migration}
    \textbf{Best-fit parameters from the Continuous-Migration model.}
    \aprcomment{fill in caption - generation time of 29 years, other details}
}
\centering
\begin{tabular}[t]{rP{8cm}cc}
    \toprule
    Parameter & Description & Value & Std. err.\\
    \midrule
    $N_e$ & Ancestral effective population size & 7270 & 1777 \\
    $N_{stem1}$ & Size of stem 1 lineage between Neanderthal and Nama splits & 8256 & 1612 \\
    $N_{stem2}$ & Size of stem 2 lineage & 13547 & 2488 \\
    $N_{Nama_0}$ & Initial Nama size & 11939 & 2989 \\
    $N_{Nama_F}$ & Final Nama size & 221 & 54 \\
    $N_{MSL_0}$ & Initial Mende size & 9738 & 2479 \\
    $N_{MSL_F}$ & Final Mende size & 28150 & 6628 \\
    $N_{EA}$ & Size of East African branch & 7489 & 1841 \\
    $N_{Gumuz_F}$ & Final Gumuz size & 3728 & 915 \\
    $N_{EP}$ & East African agriculturist size & 13072 & 3246 \\
    $N_{GBR_0}$ & Initial British size & 959 & 231 \\
    $N_{GBR_F}$ & Final British size & 11822 & 2889 \\
    $N_{Neand}$ & Neanderthal size & 2670 & 591 \\
    $T_{stems}$ & Stem split time (years) & 1163072 & 390803 \\
    $T_{Nama}$ & Nama split time (years) & 134745 & 17775 \\
    $m_{Nama-MSL}$ & Nama--Mende symmetric migration rate & $0.98\times10^{-5}$ & $0.366\times10^{-5}$ \\
    $m_{Nama-EA}$ & Nama--East Africa symmetric migration rate & $4.08\times10^{-5}$ & $1.02\times10^{-5}$ \\
    $m_{MSL-EA}$ & Mende--East Africa migration rate & $21.4\times10^{-5}$ & $5.32\times10^{-5}$ \\
    $m_{EA-GBR}$ & East Africa--Europe migration rate & $4.17\times10^{-5}$ & $1.02\times10^{-5}$ \\
    $m_{EA-EA}$ & Intra-East Africa migration rate & $33.6\times10^{-5}$ & $8.35\times10^{-5}$ \\
    $f_{GBR \rightarrow EP}$ & Ancestry proportion of East African agriculturalists from GBR 12 ka ($1-f$ from Gumuz) & 0.642 & 0.0037 \\
    $f_{EP \rightarrow Nama}$ & Ancestry proportion from EA pastoralists to Nama 2 ka & 0.255 & 0.0043 \\
    $f_{GBR \rightarrow Nama}$ & Ancestry proportion from Europeans to Nama 10 generations ago & 0.156 & 0.0021 \\
    $m_{stems}$ & Stem 1--stem 2 migration rate & $6.43\times10^{-5}$ & $1.05\times10^{-5}$ \\
    $m_{stem2-Nama}$ & Stem 2--Nama migration rate & $5.82\times10^{-5}$ & $1.60\times10^{-5}$ \\
    $m_{stem2-MSL}$ & Stem 2--Mende migration rate & $16.4\times10^{-5}$ & $4.19\times10^{-5}$ \\
    $m_{stem2-EA}$ & Stem 2--East Africa migration rate & $3.10\times10^{-5}$ & $0.901\times10^{-5}$ \\
    \bottomrule
\end{tabular}
\end{table}

\begin{table}[ht]
\caption{
    \label{tab:merger_without_stem_migration}
    \textbf{Best-fit parameters from the Merger-Without-Stem-Migration model.}
    \aprcomment{fill in caption - generation time of 29 years, other details}
}
\centering
\begin{tabular}[t]{rP{8cm}cc}
    \toprule
    Parameter & Description & Value & Std. err.\\
    \midrule
    $N_e$ & Ancestral effective population size & 11258 & 326 \\
    $N_{stem1}$ & Size of stem 1 lineage between stem 1--stem 2 split and stem 1E--stem 1S split & 113 & 76 \\
    $N_{stem2}$ & Size of stem 2 lineage & 23984 & 1149 \\
    $N_{Nama_0}$ & Initial Nama and stem 1S size & 13134 & 384 \\
    $N_{Nama_F}$ & Final Nama size & 225 & 7.3 \\
    $N_{MSL_0}$ & Initial Mende size & 11856 & 322 \\
    $N_{MSL_F}$ & Final Mende size & 25558 & 987 \\
    $N_{EA}$ & Size of East African and stem 1E branch & 9136 & 246 \\
    $N_{Gumuz_F}$ & Final Gumuz size & 3385 & 102 \\
    $N_{EP}$ & East African agriculturist size & 13650 & 408 \\
    $N_{GBR_0}$ & Initial British size & 931 & 29 \\
    $N_{GBR_F}$ & Final British size & 12064 & 334 \\
    $N_{Neand}$ & Neanderthal size & 1935 & 91 \\
    $T_{stems}$ & Stem split time (years) & 420881 & 27380 \\
    $T_{stem1}$ & Stem 1 split time into stem 1E and stem 1S (years) & 367434 & 19952 \\
    $m_{Nama-MSL}$ & Nama--Mende symmetric migration rate & $0.361\times10^{-5}$ & $0.113\times10^{-5}$ \\
    $m_{Nama-EA}$ & Nama--East Africa symmetric migration rate & $4.00\times10^{-5}$ & $0.130\times10^{-5}$ \\
    $m_{MSL-EA}$ & Mende--East Africa migration rate & $19.5\times10^{-5}$ & $0.548\times10^{-5}$ \\
    $m_{EA-GBR}$ & East Africa--Europe migration rate & $3.77\times10^{-5}$ & $0.152\times10^{-5}$ \\
    $m_{EA-EA}$ & Intra-East Africa migration rate & $37.1\times10^{-5}$ & $1.26\times10^{-5}$ \\
    $f_{GBR \rightarrow EP}$ & Ancestry proportion of East African agriculturalists from GBR 12 ka ($1-f$ from Gumuz) & 0.647 & 0.0037 \\
    $f_{EP \rightarrow Nama}$ & Ancestry proportion from EA pastoralists to Nama 2 ka & 0.257 & 0.0042 \\
    $f_{GBR \rightarrow Nama}$ & Ancestry proportion from Europeans to Nama 10 generations ago & 0.156 & 0.0021 \\
    $T_{Nama}$ & Time of Nama merger event & 117392 & 8253 \\
    $f_{stem 2 \rightarrow Nama}$ & Proportion of stem 2 ancsestry making up initial Nama lineage ($1-f$ from stem 1S) & 0.707 & 0.0086 \\
    $T_{EA}$ & Time of East Africa merger event & 94892 & 3648 \\
    $f_{stem 2 \rightarrow EA}$ & Proportion of stem 2 ancestry making up initial East Africa lineage ($1-f$ from stem 1E) & 0.481 & 0.0074 \\
    $T_{MSL}$ & Time of secondary admixture from stem 2 to Mende & 23922 & 570 \\
    $f_{stem 2 \rightarrow MSL}$ & Proportion of ancestry from secondary stem 2 admixture to Mende & 0.168 & 0.0036 \\
    \bottomrule
\end{tabular}
\end{table}

\begin{table}[ht]
\caption{
    \label{tab:merger_with_stem_migration}
    \textbf{Best-fit parameters from the Merger-With-Stem-Migration model.}
    \aprcomment{fill in caption - generation time of 29 years, other details}
}
\centering
\begin{tabular}[t]{rP{8cm}cc}
    \toprule
    Parameter & Description & Value & Std. err.\\
    \midrule
    $N_e$ & Ancestral effective population size & 11479 & 1369 \\
    $N_{stem1}$ & Size of stem 1 lineage between Neanderthal split and stem 1E--stem 1S split & 117 & 838 \\
    $N_{stem2}$ & Size of stem 2 lineage & 24393 & 6668 \\
    $N_{Nama_0}$ & Initial Nama size & 13211 & 1514 \\
    $N_{Nama_F}$ & Final Nama size & 223 & 31 \\
    $N_{MSL_0}$ & Initial Mende size & 11444 & 1165 \\
    $N_{MSL_F}$ & Final Mende size & 27417 & 4332 \\
    $N_{EA}$ & Size of East African Branch & 9077 & 1628 \\
    $N_{Gumuz_F}$ & Final Gumuz size & 3402 & 337 \\
    $N_{EP}$ & East African agriculturist size & 13506 & 1684 \\
    $N_{GBR_0}$ & Initial British size & 953 & 122 \\
    $N_{GBR_F}$ & Final British size & 12406 & 1678 \\
    $N_{Neand}$ & Neanderthal size & 2416 & 235 \\
    $T_{stems}$ & Stem split time (years) & 1442022 & 426449 \\
    $T_{stem1}$ & Stem 1S--stem 1E split time (years) & 479401 & 166339 \\
    $m_{Nama-MSL}$ & Nama--Mende symmetric migration rate & $0.712\times10^{-5}$ & $0.401\times10^{-5}$ \\
    $m_{Nama-EA}$ & Nama--East Africa symmetric migration rate & $4.35\times10^{-5}$ & $0.912\times10^{-5}$ \\
    $m_{MSL-EA}$ & Mende--East Africa migration rate & $19.8\times10^{-5}$ & $2.57\times10^{-5}$ \\
    $m_{EA-GBR}$ & East Africa--Europe migration rate & $3.87\times10^{-5}$ & $0.550\times10^{-5}$ \\
    $m_{EA-EA}$ & Intra-East Africa migration rate & $35.9\times10^{-5}$ & $5.36\times10^{-5}$ \\
    $f_{GBR \rightarrow EP}$ & Ancestry proportion of East African agriculturalists from GBR 12 ka ($1-f$ from Gumuz) & 0.640 & 0.0075 \\
    $f_{EP \rightarrow Nama}$ & Ancestry proportion from EA pastoralists to Nama 2 ka & 0.257 & 0.0049 \\
    $f_{GBR \rightarrow Nama}$ & Ancestry proportion from Europeans to Nama 10 generations ago & 0.157 & 0.0031 \\
    $m_{stems}$ & Stem 1--stem 2 migration rate & $11.6\times10^{-5}$ & $8.74\times10^{-5}$ \\
    $T_{Nama}$ & Time of Nama merger event & 118547 & 28170 \\
    $f_{stem 2 \rightarrow Nama}$ & Proportion of stem 2 ancsestry making up initial Nama lineage ($1-f$ from stem 1S) & 0.714 & 0.067 \\
    $T_{EA}$ & Time of East Africa merger event & 98083 & 8865 \\
    $f_{stem 2 \rightarrow EA}$ & Proportion of stem 2 ancestry making up initial East Africa lineage ($1-f$ from stem 1E) & 0.495 & 0.059 \\
    $T_{MSL}$ & Time of secondary admixture from stem 2 to Mende & 25119 & 641 \\
    $f_{stem 2 \rightarrow MSL}$ & Proportion of ancestry from secondary stem 2 admixture to Mende & 0.181 & 0.0085 \\
    \bottomrule
\end{tabular}
\end{table}

\clearpage

\addcontentsline{toc}{section}{Supporting figures}
\section*{Supporting figures}

\end{document}
